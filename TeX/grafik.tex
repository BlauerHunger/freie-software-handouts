\include{header}

\begin{document}
\frontpage{Grafik \& \\ Bildbearbeitung} % XXX

\software{JPEGView}{Bildbetrachter}
{Einfacher Betrachter für Bilder in den gängigen Formaten mit Diaschaufunktion und einfachen Bearbeitungsmöglichkeiten.}
{Windows}
{http://jpegview.sourceforge.net}
{GPL}
{nein}

\software{Gimp}{Bildbearbeitung}
{Mächtiges Bildbearbeitungsprogramm mit großem Funktionsumfang. Bekannteste freie Alternative zu Photoshop.}
{GNU/Linux, Windows, Mac u.\,a.}
{http://www.gimp.org}
{GPL}
{ja}

\software{Krita}{Bildbearbeitung}
{Mächtiges Bildbearbeitungs- und Malprogramm.}
{GNU/Linux, Windows, Mac u.\,a.}
{https://krita.org}
{GPL}
{nur unter Linux}

\software{Inkscape}{Vektorzeichenprogramm}
{Vektorgrafik-Programm - arbeitet mit SVG-Dateien, unterstützt Import und Export von PDF-Dateien und zahlreichen anderen Formaten.}
{GNU/Linux, Windows, Mac u.\,a.}
{http://www.inkscape.org/de/}
{GPL}
{ja}


\software{Hugin}{Panorama}
{Umfangreiche Software zu Erstellung von Panoramen aus mehreren Einzelaufnahmen.}
{GNU/Linux, Windows, Mac u.\,a.}
{http://hugin.sourceforge.net}
{GPL}
{ja}

\software{RawTherapee}{RAW-Konverter}
{Mit RawTherapee können sie die Rohdaten ihrer Kamera in Bilder umwandeln und dabei bearbeiten.}
{GNU/Linux, Windows, Mac}
{http://rawtherapee.com}
{GPL}
{ja}

\software{MyPaint}{Zeichenprogramm}
{Zeichenprogrammm für intuitives Malen.}
{GNU/Linux, Windows, Mac u.\,a.}
{http://mypaint.org/}
{GPL}
{ja}

\software{Darktable}{Foto-Workflow-Software}
{Virtueller Leuchttisch und Dunkelkammer für Fotografen mit Funktionen zur Bildbearbeitung und -verwaltung.}
{GNU/Linux, Mac u.\,a.}
{http://www.darktable.org}
{GPL}
{ja}

\backpage

\end{document}
