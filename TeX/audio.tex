\include{header}


\begin{document}

\frontpage{Musik \& \\ Tonbearbeitung}


\software{Nightingale}{Audioplayer}
{Erlaubt die Organisation und Wiedergabe von Musikdateien. Bietet zahlreiche Erweiterungsmöglichkeiten, wie z.B. die Einbindung von Internetradios.}
{GNU/Linux, Windows, Mac}
{https://getnightingale.com}
{GPL, MPL, BSD}
{ja}

\software{Amarok}{Audioplayer}
{Erlaubt die Organisation und Wiedergabe von Musikdateien bei vielfältigem Funktionsumfang. Bestandteil des KDE-Projekts.}
{GNU/Linux, Windows, Mac}
{https://amarok.kde.org/de}
{GPL}
{ja}

\software{Tomahawk}{Audioplayer}
{Verfügt über Anbindungen zu zahlreichen webbasierten Musikdiensten und kann, neben lokalen Musikdateien, deren Inhalte nahtlos abspielen.}
{GNU/Linux, Windows, Mac u.\,a.}
{https://www.tomahawk-player.org}
{GPL}
{ja}

\software{Performous}{Musikspiel}
{Karaoke-Party-Spiel mit Einblendung von Text, Tonhöhe und -länge.}
{GNU/Linux, Windows, Mac}
{http://performous.org} %Weiterleitung zu https://performous.org mit github SSL Zertifikat und HTTP Strict Transport Security (HSTS) 
{GPL}
{ja}


\software{Audacity}{Audioeditor}
{Ermöglicht die Aufnahme, Bearbeitung (Mischen, Schneiden u.a.) und Optimierung (z.B. Reduktion von Störgeräuschen) von Audiodateien.}
{GNU/Linux, Windows, Mac}
{http://www.audacityteam.org/} %Kein https://
{GPL}
{ja}

\software{LilyPond}{Notensatzprogramm}
{Erlaubt Notendruck auf hohem ästhethischen Niveau. Kann auch über grafische Notensatzprogramme wie Denemo oder MuseScore genutzt werden.}
{GNU/Linux, Windows, Mac u.\,a.}
{http://lilypond.org} %Kein https://
{GPL}
{teilweise}

\software{Mixxx}{DJ-Programm}
{Bietet virtuelle Plattenspieler mit Tonhöhenregelung und automatischem Beatmatching für DJs sowie eine Auto-DJ-Funktionalität.}
{GNU/Linux, Windows, Mac}
{https://mixxx.org}
{GPL}
{ja}

\software{Rosegarden}{Midisequenzer}
{Kompositions- und Schnittprogramm für mehrspurige Midi-Dateien.}
{GNU/Linux}
{http://www.rosegardenmusic.com} %Kein https://
{GPL}
{ja}

\backpage

\end{document}
