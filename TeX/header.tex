\documentclass[12pt,a4paper]{leaflet}
\usepackage[utf8]{inputenc}
\usepackage[german]{babel}
\usepackage{microtype}
\usepackage{graphicx}
\usepackage{color}
\definecolor{gray}{gray}{.5}
\setlength{\parindent}{0pt} 
\usepackage{blindtext}
\usepackage[condensed]{tgheros}
\setmargins{2cm}{1cm}{8mm}{8mm}

\usepackage{hyperref}
\hypersetup{
    pdftoolbar=true,
    pdfmenubar=true,
    pdffitwindow=false,
    pdfstartview={FitH},
    pdfnewwindow=true,
    colorlinks=true,
    linkcolor=black,
    citecolor=black,
    filecolor=black,
    urlcolor=black
}
\def\UrlFont{\bfseries}
\pagestyle{empty} 

\newcommand{\software}[7]{
\begin{minipage}[t][0.4\textheight]{1\textwidth}
\flushleft
\subsection*{#1 \hfill \textcolor{gray}{#2}}
#3 
\medskip

%\scalebox{.9}[1.0]{\textit{Betriebssystem:} \textbf{#4}} \\
\textit{Betriebssystem:} \textbf{#4} \\
\textit{Lizenz:} \textbf{#6} \\
\textit{Deutschsprachig:} \textbf{#7} \\
\textit{Homepage:} \url{#5} \\
\vglue  2 true cm
\end{minipage}
}

\newcommand{\frontpage}[1]{
\begin{titlepage}
\begin{flushleft}
\vglue  2 true cm

\textsc{\LARGE Freie Software} 
\newcommand{\HRule}{\rule{0.9\linewidth}{0.5mm}}
\HRule \\[0.4cm]
\textsc{\textbf{\LARGE #1}}

\vglue  2 true cm

{ \large\it
Acht Freie \\
Computerprogramme, \\
die Deine Rechte \\
als Nutzer/-in \\
respektieren \\
und schützen. }
\end{flushleft}
\vfill
\end{titlepage}

\flushleft
}

\newcommand{\backpage}{
\newpage
\subsection{Was ist »Freie Software«?}
Von »Freier Software« spricht man, sobald ein
Software-Hersteller sein \textbf{Computerprogramm} unter
einer \textbf{Freien Softwarelizenz} veröffentlicht.

Im Gegensatz zu proprietären Softwarelizenzen
(welche die Art und Weise der Software-Nutzung meist
zugunsten des Herstellers einschränken), zielen \textbf{Freie
Softwarelizenzen} darauf ab, \textbf{Deine Rechte als
Nutzer/-in} zu verteidigen. Diese Rechte können
anhand von \textbf{vier Freiheiten} beschrieben werden:
\begin{itemize}
\item Der Freiheit, das Programm jederzeit und für jeden
beliebigen Zweck zu \textbf{verwenden}
\item Der Freiheit, die Funktionsweise des Programms zu
\textbf{verstehen}, indem Du seinen Programmiercode
untersuchst
\item Der Freiheit, das Programm zu \textbf{verändern}, um es
Deinen Anforderungen und Wünschen anzupassen
\item Der Freiheit, Kopien des (un)veränderten
Programms an Deine Mitmenschen zu \textbf{verteilen},
damit auch Andere davon profitieren können
\end{itemize}

Weitere Informationen zu Freier Software, Freien
Softwarelizenzen und dem »Free Software Movement«
findest Du unter \url{www.fsfe.org}.
}
